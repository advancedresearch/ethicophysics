\documentclass{article}
\usepackage{leonine,amsmath,amssymb,amsthm,graphicx}
\setkeys{Gin}{width=\linewidth,totalheight=\textheight,keepaspectratio}
\graphicspath{{graphics/}}
% Prints a trailing space in a smart way.
\usepackage{xspace}
% Inserts a blank page
\newcommand{\blankpage}{\newpage\hbox{}\thispagestyle{empty}\newpage}
% \usepackage{units}
% Typesets the font size, leading, and measure in the form of 10/12x26 pc.
\newcommand{\measure}[3]{#1/#2$\times$\unit[#3]{pc}}

\theoremstyle{definition}
\newtheorem{pred}[thm]{Prediction}

\title{Askesis: Introduction} \author{Eric Purdy}

\begin{document}

\maketitle

\begin{itemize}
\item Audience - neuroscientists at this point
\item What's accepted/known vs. what I'm doing
\item Where is this coming from - applying modern machine learning
  techniques
\item Terminology is new
\item We propose novel capabilities that emerge from known connections
  and learning mechanisms. These novel capabilities are plausible
  candidates for the function of the cerebellum. The emergence of the
  capabilities is in keeping with well-understood machine learning
  research.
\item Point out where divergence happens
\item Spell out relationship between neuroscience and machine learning
  aspects of this
\item We plan to implement these algorithms and use them to solve
  problems in machine learning and robotics.
\end{itemize}


\end{document}
