\documentclass{article}
\usepackage{leonine,amsmath,amssymb,amsthm,graphicx}
\setkeys{Gin}{width=\linewidth,totalheight=\textheight,keepaspectratio}
\graphicspath{{graphics/}}
% Prints a trailing space in a smart way.
\usepackage{xspace}
% Inserts a blank page
\newcommand{\blankpage}{\newpage\hbox{}\thispagestyle{empty}\newpage}
% \usepackage{units}
% Typesets the font size, leading, and measure in the form of 10/12x26 pc.
\newcommand{\measure}[3]{#1/#2$\times$\unit[#3]{pc}}

\theoremstyle{definition}
\newtheorem{pred}[thm]{Prediction}

\title{Askesis: Negative Pathway} \author{Eric Purdy}

\begin{document}

\maketitle


\subsection{Granular cells encode sets of mossy fibers}

Granular cells are the most numerous cells in the human brain; 3/4 of
the neurons in the brain are granular cells in the cerebellum. This
number suggests that they must embody some sort of combinatorial
explosion, which is a feature found in both Marr's and Albus's models. 

Each granular cell has synapses with 4-5 mossy fibers, but not all of
these synapses are necessarily active. We posit that each granular
cell learns a subset of their inputs that co-occur most frequently,
and subsequently fire only when all of those inputs are active. We
will call this subset the active set of the granular cell.

In the case that all the mossy fibers in the active set arise from
muscle spindles, a given granular cell will encode a particular set of
muscle lengths, which will in some cases suffice to specify the
angular location of some joint, or even several joints. In other
cases, it is still a potentially useful feature to feed to the
classifier.

We postulate that the granular cell learns the active subset by
looking for co-occurrences of its inputs. There are two mechanisms
available to the granular cell: it can strengthen or weaken the mossy
fiber-granular cell synapse, and it can raise or lower the threshold
for the granular cell to fire (``intrinsic plasticity''). 

In order to be biologically feasible, the strengthening or weakening
of the mossy fiber-granular cell synapse should be based solely on
information that is locally available. Fortunately, these synapses
live inside of glomeruli which also receive input from Golgi cells, so
we can use the firing of the Golgi cell when we decide whether to
strengthen or weaken the synapse.

As we will discuss below, Golgi cells can be fired either by the
firing of several nearby (and thus likely to be related) mossy fibers,
or by the firing of some number of nearby granular cells. Of interest
here is the first mechanism; if we know that several nearby mossy
fibers have fired, then it is likely that the ``true'' active set is
firing.  By ``true'' active set, we means the largest set of mossy
fibers that fire together more than a certain threshold amount of the
time. If a particular mossy fiber is active at the same time as the
true active set (as signified by the firing of the Golgi cell), then
it is probably a member of the active set, so we should strengthen the
synapse between the mossy fiber and the granular cell. If the mossy
fiber is not active at the same time, we should weaken the synapse.
This can be accomplished by using something like the BCM rule, except
that we should use the firing of the Golgi cell in place of the
post-synaptic activation.

Let $G^{init}_i(t)$ be the fraction of the time that the $i$-th
granular cell is firing during a small time window ending at time $t$,
so that $G^{init}_i(t)=1$ when the $i$-th cell is firing at its
maximum rate. We can assume that the timings of the individual mossy
fibers are independent and random, i.e., given the rate at which each
is firing, and given that the granular cell only fires when all of its
active inputs fire, we can assume that the rates simply
multiply:
$$G^{init}_i(t) = \prod_{j\in \AAA_i} M_j(t),$$
where $\AAA_i$ is the active set of the $i$-th granular cell.

\subsection{Timed non-maximal suppression from the Golgi cells}

It will commonly be the case that two granular cells have very similar
input patterns. For instance, two granular cells can be excited by the
same mossy fiber, or by two different mossy fibers with similar firing
patterns. In particular, mossy fibers that ultimately encode firings
of muscle spindles will be highly correlated if they encode similar
lengths of the same muscle. Since granular cells are triggered by sets
of mossy fibers, this means that two granular cells could also wind up
being highly correlated.

It is therefore desirable to perform what is called non-maximal
suppression. Among a set of granular cells that fire in similar
situations, we want to select the granular cell that is firing the
most rapidly, and suppress the firing of all the other similar
granular cells. A particular granular cell is active only if it is the
one under its Golgi cell that is firing most rapidly, which indicates
that the state it codes for is the closest to the truth. 

\subsection{Purkinje cells as difference of post-Golgi granular images}

The Purkinje cells receive input from the granular cells. These
synapses are in the molecular layer, the outermost layer of the
cerebellum. Each Purkinje cell receives input from something like
100,000 granular cells.

Each Purkinje cell receives in addition a single input from a
``climbing fiber'', which is the output of a neuron in the inferior
olive. This input is strong enough to trigger the Purkinje cell to
fire. It has been established that the synapse between the output of
the granular cell and the input of the Purkinje cell is weakened if
the granular cell fires at the same time that the climbing fiber
fires; this is called long-term depression. Our theory is that the
Purkinje cell firing is meant to suppress some specific output; the
long-term depression is desirable because it means that we will
henceforth not take the firing of a particular granular cell as a
reason to fire the Purkinje cell and suppress an output that we desire
to see remain unsuppressed.

We should think of each Purkinje cell as receiving a ``picture'' or
``video'' of what is going on from the granular cells. The Purkinje
cells are receiving as much information as a 300x300 binary image.

$$P_j(t) = \begin{cases} 1 & \sum_j W^{pp}_{ij}(t) G^{nms}_{i}(t) > \theta_j \\
0 & \mbox{otherwise} \end{cases},$$
where
\begin{itemize}
\item $G^{nms}_i(t)$ is the output of the $i$-th granular cell at time $t$,
  after non-maximal suppression has been performed by the Golgi cells.
\item $W^{pp}_{ij}(t)$ is the weight of the synapse between the $i$-th
  parallel fiber (output of the $i$-th granular cell) and the $j$-th
  Purkinje cell.
\item We can probably get away with $\theta_j=\theta$, for some
  constant $\theta$.
\end{itemize}


We posit the following learning rule for the synapse weights between
the parallel fibers (output of the granular cells) and Purkinje cell
inputs:
$$W^{pp}_{ij}(t+1) = \begin{cases} 
W^{pp}_{ij}(t) - c & IO_j(t)=1, G_{i}(t)=1 \\
W^{pp}_{ij}(t) + d & IO_j(t)=0, G_i(t)=1 \\
W^{pp}_{ij}(t) & G_i(t)=0 \end{cases}.$$

Given this learning rule, we can see that the synapse weights of a
single Purkinje cell can be thought of as the sum of all post-Golgi
granular images from time steps where the climbing fiber did not fire,
minus the sum of all post-Golgi granular images from time steps where
the climbing fiber did fire:
\begin{align*}
W^{pp}_{ij}(T) &= d\sum_{t=1}^{T-1} \left(1-IO_j(t)\right) G^{nms}_i(t) \\
& - c \sum_{t=1}^{T-1} IO_j(t) G^{nms}_i(t)
\end{align*}
We can think of this in the following way: every time step of training
is either anti-firing of the Purkinje cell (if the relevant inferior
olive cell fired at that time step) or pro-firing (if the relevant
inferior olive cell did not fire). We exclude all time steps when
$G^{nms}_i(t)$ was zero. We then allow each time step to vote at this
synapse, weighting the votes against firing with weight $c$ and the
votes in favor of firing with weight $d$. We perform this computation
at each synapse to get the synapse weight $W^{pp}_{ij}$, which is then
used as the weight with which $G^{nms}_i(t)$ is considered. Thus, the
Purkinje cell takes each $G^{nms}_i(t)$ into account with a weight
that depends on all previous time steps where that cell was active.

\subsection{Basket and stellate cells as anti-Purkinje cells}

We have a problem here, which is that the synapses between the
parallel fibers and the Purkinje cells are always excitatory, so we
can never take a granular cell as evidence that we {\em don't} want to
fire the relevant Purkinje cell. This problem is solved by the basket
and stellate cells, which inhibit the Purkinje cells and receive their
input from the parallel fibers. Whenever we would wind up with a
negative synapse weight between the $i$-th granular cell and the
$j$-th Purkinje cell in the above formulas, this should be interpreted
as a positive weight between the parallel fiber of the $i$-th granular
cell and a basket or stellate cell that inhibits the $j$-th Purkinje cell.



\section{Thoughts}

\begin{itemize}
\item Can think of NMS as a kind of explaining away.
\item Why are there mossy fiber-Golgi cell synapses?
\item Purkinje cells have collaterals that suppress nearby Purkinje
  cells - this is sufficient for symmetry breaking?
\item Purkinje cells have collaterals that have ``weak inhibitory
  synapses'' with cortical inhibitory interneurons (B/S cells)

\item There are some connections between collaterals of granular cells
  and Purkinje cells that take place in the Purkinje layer. These are
  stronger synapses than those of the parallel fibers. This is the
  symmetry breaking mechanism.

\item Golgi cells have gap junctions with other Golgi cells, which
  seem to act to synchronize them.

\item (Prediction already known to be true.)
The parallel fiber-basket cell synapse should be strengthened by the
simultaneous firing of the climbing fiber and the parallel fiber, and
weakened by the firing of the parallel fiber alone.


\end{itemize}


\end{document}
