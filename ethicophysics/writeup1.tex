\documentclass{article}
\usepackage{amsmath, amssymb, amsthm, leonine}

\title{Ethicophysics I}

\author{Eric Purdy}

\begin{document}

\maketitle

\begin{abstract}
\end{abstract}

\section{Introduction}

In this document, we lay out the beginnings of a new theory of human
nature that we term {\em ethicophysics}. This is intended to be a
complete and scientifically accurate account of the nature of Good and
Evil, and other such ethical riddles that have haunted humanity since
the beginning of our species. We term it ethicophysics to suggest that
there are certain natural laws in the ethical sphere that cannot be
violated any more than the laws of physics can be violated.

Since such a project is ambitious to the point of madness, we ask the
reader's indulgence in following along with what must seem a quixotic
quest to end all quixotic quests. Nevertheless, we hold that some
things are true and some things are false, that some actions are good
and some are evil. Ultimately, words mean things, not because the
universe says they must, but because we choose to use them in a
certain way and not in other ways.

\section{On God and Souls}

We use the term ``God'' to refer to a potential omniscient observer of
the universe. We make no claims as to the ontological status of such a
being. Note, in particular, that we do not assume that God is
omnipotent or omnibenevolent, which allows us to avoid the classic
Epicurean trilemma:

\begin{quote}
  God, he says, either wishes to take away evils, and is unable; or He
  is able, and is unwilling; or He is neither willing nor able, or He
  is both willing and able. If He is willing and is unable, He is
  feeble, which is not in accordance with the character of God; if He
  is able and unwilling, He is envious, which is equally at variance
  with God; if He is neither willing nor able, He is both envious and
  feeble, and therefore not God; if He is both willing and able, which
  alone is suitable to God, from what source then are evils? Or why
  does He not remove them?\cite{trilemma}
\end{quote}

We note, however, that the content of the ethicophysics suggests that
such an entity, if it did exist, would be reasonably omnibenevolent,
and as omnipotent as is consistent with the existence of free will.
As noted by Dr. Martin Luther King Jr. \cite{king1963tough}, a God that
did not allow for free will would simply be a tyrant.

\subsection{Defining the Soul}

We define the soul of an individual actor to be {\em that which is
  true about the actor}. In religious terms, it is basically God's
opinion about the actor.

Note that, in particular, that which is true about the actor includes
what opinion every human that ever lived would have of the actor if
they were given true knowledge of the events and choices of that
actor's existence. This sort of ``subjective truth'' will be deeply
contradictory (presumably e.g. Hitler and Churchill would disagree
about a lot), but it is no less real for that.

\subsection{On the Equality of Souls}

Many have noted that one can choose to view all human beings as
fundamentally equal in the context of ethics, e.g.:

\begin{quote}
  Do to others what you want them to do to you. This is the meaning of
  the law of Moses and the teaching of the prophets. \cite{mount}
\end{quote}

\begin{quote}
  We hold these Truths to be self-evident, that all Men [sic] are
  created equal, that they are endowed by their Creator with certain
  unalienable Rights, that among these are Life, Liberty, and the
  pursuit of Happiness... \cite{independence}
\end{quote}

If we expand this slightly to include all actors that have both a soul
and a mind, it seems as good a foundation as any for a theory of
ethics. In particular, given our definition of the soul, any actor
with a mind can be said to have a soul. This includes, in our opinion,
animals (see e.g. \cite{singer1995animal, coetzee1999lives}) and
sufficiently advanced computer programs (see
\cite{turing1954computing}).

\section{Main Results}

In this section, we pursue traditional mathematical proofs of certain
propositions in the field of ethics.

\begin{thm}
  Actions have consequences. In particular, the consequence of
  committing an evil act that goes undetected is that one becomes the
  person that one becomes after such an act, and has as a consequence
  much less self-respect.
\end{thm}

\begin{proof}
  Note that this proof needs to be checked over very thoroughly, as it
  may contain errors.

  Consider the Lagrangian $\LLL(q, \dot{q}, t) = \TTT - \UUU$, where
  $\TTT$ is the kinetic energy of a system, and $\UUU$ is the
  potential energy of that system. Here $q$ is the physical state of
  the system in generalized coordinates.

  Let $\SSS$ be the ``subjective energy'' of the system. This can be
  any quantity of interest that depends on the generalized coordinates
  $q, \dot{q}, t$ of the system and the ``subjective coordinates'' $s,
  \dot{s}, t$ (which are supposed to have no physical realization that
  is legible to the laws of physics), and which is held to be
  conserved. Note that we will generally be more interested in the
  ``subjective history'' function
  $$\HHH(t) = \int_0^t \SSS(q, \dot{q}, s, \dot{s}, t).$$

  Let $\tau(t)$ (called the ``tweak'') be a continuous symmetry of the
  physical system, i.e., for infinitesimal $\epsilon$, the
  transformation
  \begin{align*}
    q(t) &\to q(t) + \epsilon \tau(t) \\
    \dot{q}(t) &\to \dot{q}(t) + \epsilon \dot{\tau}(t) \\
  \end{align*}
  leaves the Lagrangian unaffected.

  Let $\varphi(s)$ (called the ``flip'') be a discrete non-physical
  symetry of the subjective energy function at time $t_{\varphi}$,
  i.e., a function such that, for one brief instant of time,
  $$\SSS(q, \dot{q}, \varphi(s), \frac{d}{dt} \varphi(s), t_{\varphi}) =
  \SSS(q, \dot{q}, s, \dot{s}, t_{\varphi}).$$

  Consider a finite time interval $[0, T]$ corresponding to some
  meaningful time period (e.g., the lifetime of a ethicophysical
  actor). Define the following quantity (the ``God Lagrangian''):
  $$\GGG(q, \dot{q}, s, \dot{s}, t) = \LLL(q, \dot{q}, t) +
  \SSS(q, \dot{q}, s, \dot{s}, t) - \SSS(q, \dot{q}, \varphi(s), \frac{d}{dt}\varphi(s), t)$$

  By Noether's Theorem, the following quantity is conserved:
  $$\sum_{i=1}^n \frac{\partial \LLL}{\partial \dot{q}_i} \tau_i.$$

  By Noether's Theorem applied to the modified Lagrangian, the same is
  true of the quantity
  $$\sum_{i=1}^n \frac{\partial \GGG}{\partial \dot{q}_i} \tau_i +
  \sum_{i=1}^m \frac{\partial \GGG}{\partial \dot{s}_i} \varphi_i$$

  After doing some algebra, we arrive at the conclusion that the
  following quantity is conserved by the laws of ethicophysics:
  $$\sum_{i=1}^n \frac{\partial \SSS}{\partial \dot{q}_i} \tau_i -
  \sum_{i=1}^n \frac{\partial \SSS_\varphi}{\partial \dot{q}_i} \tau_i +
  \sum_{i=1}^m \frac{\partial \SSS}{\partial \dot{s}_i} \varphi_i -
  \sum_{i=1}^m \frac{\partial \SSS_\varphi}{\partial \dot{s}_i} \varphi_i,
  $$
  which simplifies to the equation
  $$\sum_{i=1}^n \frac{\partial \SSS}{\partial \dot{q}_i} \tau_i +
  \sum_{i=1}^m \frac{\partial \SSS}{\partial \dot{s}_i} \varphi_i =
  \sum_{i=1}^n \frac{\partial \SSS_\varphi}{\partial \dot{q}_i} \tau_i +
  \sum_{i=1}^m \frac{\partial \SSS_\varphi}{\partial \dot{s}_i} \varphi_i,
  $$

  We are now ready to finish the proof. Consider some binary decision
  that can be made, and consider the two possible timestreams that
  will follow making either choice. Let $\SSS = s$ be the quantity of
  self-respect that one feels with respect to the decision in
  question. (It is presumed that this can be measured precisely in
  some way via e.g. advanced neuroscientific theories that we do not
  presume to know. The important point here is just that there will
  come to exist some rigorous technical definition of the quantity
  such that its epistemological status is not in doubt.)

  Suppose, further, that the decision has no consequences that are
  perceivable in the external physical world after some time period
  $t_{\mathrm{hide the body}}$ has elapsed. Thus, after this point,
  $\SSS$ ``should'' no longer depend on $q$.

  There is then an additional conserved quantity (in addition to
  $\SSS$), which is the ``respect momentum''
  $$\frac{\partial S_\varphi}{\partial s} s.$$


  Then we arrive at a very simple law:

  \begin{align*}
    \HHH(T) &= \HHH(t_{\mathrm{hide the body}}) +
    \int_{t_{\mathrm{hide the body}}}^T \frac{\partial{\SSS}}{\partial \dot{s}} ds \\
    &= \HHH(t_{\mathrm{hide the body}}) +
    \int_{t_{\mathrm{hide the body}}}^T \frac{\partial{\SSS_\varphi}}{\partial \dot{s}} ds \\
  \end{align*}

  Thus, the total lifetime integral of respect momentum will be the
  same in either case (i.e., in both timestreams). But, assuming the
  decision is one with a clear right answer, the predominant sign of
  $\frac{\partial \SSS}{\partial s}$ will be the opposite of the
  predominant sign of $\frac{\partial \SSS_\varphi}{\partial
    s}$. Thus, making the wrong decision will have hugely negative
  consequences for one's self-respect, as expected.
  
\end{proof}

\bibliography{ethicophysics} 
\bibliographystyle{acm}


\end{document}
