\documentclass{article}
\usepackage{amsmath, amssymb, amsthm, leonine}

\title{Ethicophysics I}

\author{Eric Purdy}

\begin{document}

\maketitle

\begin{abstract}
What are Good and Evil? How do we explain these concepts to a computer
sufficiently well that we can be assured that the computer will
understand them in the same sense as humans understand them? These are
hard questions, and people have often despaired of finding any answers
to the AI safety problem.

In this paper, we lay out a theory of ethics modeled on the laws of
physics. The theory has two key advantages: it squares nicely with
most human moral intuitions, and it is amenable to rather
straightforward computations that a computer could easily perform if
told to. It therefore forms an ideal foundation for solving the AI
safety problem.
\end{abstract}

\section{Introduction}

In this document, we lay out the beginnings of a new theory of ethics
and human nature that we term {\em ethicophysics}. This is intended to
be a complete and scientifically accurate account of the nature of
Good and Evil, and other such ethical riddles that have haunted
humanity since the beginning of our species. We term it ethicophysics
to suggest that there are certain natural laws in the ethical sphere
that cannot be violated any more than the laws of physics can be
violated.

Since such a project is ambitious to the point of madness, we ask the
reader's indulgence in following along with what must seem a quixotic
quest to end all quixotic quests. Nevertheless, we hold that some
things are true and some things are false, that some actions are good
and some are evil. Ultimately, words mean things, not because the
universe says they must, but because we choose to use them in a
certain way and not in other ways.

We consider an {\em actor network}, which is a set of actors who act
in the same physical space and communicate with one another. The minds
of the actors are presumed to be {\em non-physical}, i.e., they are
powered by computational devices which are not modeled by the laws of
physics used to reason about the rest of reality. This is obviously a
weird assumption - all really existing computational devices (brains,
computers, abacuses, etc.) are physical and obey the physical laws of
reality. The goal here is to separate reality out into the {\em naive}
physical reality modeled by traditional physics and the {\em ethical}
physical reality modeled by the ethicophysics. Since computational
devices exist in reality and have the properties they have because of
the laws of physical reality, the ethicophysics is in some sense a
proper subset of ``real'' physics; thus ethicophysics and traditional
physics coexist as partners in describing the laws of reality, rather
than fighting one another.

It is presumed that actors can communicate ideas to one another at
will through non-physical means; this is again a strange assumption,
but we make it for similar reasons as above.

It is not presumed that actors are virtuous, ethical, truthful,
etc. In fact, the predominant motivating question in the ethicophysics
is why people aren't significantly more evil than they appear to be.

\section{On God and Souls}

We use the term ``God'' to refer to a potential omniscient observer of
the universe. We make no claims as to the ontological status of such a
being. Note, in particular, that we do not assume that God is
omnipotent or omnibenevolent, which allows us to avoid the classic
Epicurean trilemma \cite{trilemma}:

\begin{quote}
  God, he says, either wishes to take away evils, and is unable; or He
  is able, and is unwilling; or He is neither willing nor able, or He
  is both willing and able. If He is willing and is unable, He is
  feeble, which is not in accordance with the character of God; if He
  is able and unwilling, He is envious, which is equally at variance
  with God; if He is neither willing nor able, He is both envious and
  feeble, and therefore not God; if He is both willing and able, which
  alone is suitable to God, from what source then are evils? Or why
  does He not remove them?
\end{quote}

We note, however, that the content of the ethicophysics suggests that
such an entity, if it did exist, would be reasonably omnibenevolent,
and as omnipotent as is consistent with the existence of free will.
As noted by Dr. Martin Luther King Jr. \cite{king1963tough}, a God that
did not allow for free will would simply be a tyrant:
\begin{quote}
  I am thankful that we worship a God who is both tough minded and
  tenderhearted. If God were only tough minded, he would be a cold,
  passionless despot sitting in some far-off heaven ``contemplating
  all,'' as Tennyson puts it in ``The Palace of Art.'' He would be
  Aristotle's ``unmoved mover,'' self-knowing but not other-loving.
  But if God were only tenderhearted, he would be too soft and
  sentimental to function when things go wrong and incapable of
  controlling what he has made. He would be like H.G. Wells's loveable
  God in {\em God, the Invisible King}, who is strongly desirous of
  making a good world but finds himself helpless before the surging
  powers of evil. God is neither hardhearted nor soft minded. He is
  tough minded enough to transcend the world; he is tenderhearted
  enough to live in it. He does not leave us alone in our agonies and
  struggles. He seeks us in dark places and suffers with us and for us
  in our tragic prodigality.
\end{quote}

\subsection{Defining the Soul}

We define the soul of an individual actor to be {\em that which is
  true about the actor}. In religious terms, it is basically God's
opinion about the actor.

Note that, in particular, that which is true about the actor includes
what opinion every human that ever lived would have of the actor if
they were given true knowledge of the events and choices of that
actor's existence. This sort of ``subjective truth'' will be deeply
contradictory (presumably e.g. Hitler and Churchill would disagree
about a lot), but it is no less real for that.

\subsection{On the Equality of Souls}

Many have noted that one can choose to view all human beings as
fundamentally equal in the context of ethics, e.g.:

\begin{quote}
  Do to others what you want them to do to you. This is the meaning of
  the law of Moses and the teaching of the prophets. \cite{mount}
\end{quote}

\begin{quote}
  We hold these Truths to be self-evident, that all Men [sic] are
  created equal, that they are endowed by their Creator with certain
  unalienable Rights, that among these are Life, Liberty, and the
  pursuit of Happiness... \cite{independence}
\end{quote}

If we expand this slightly to include all actors that have both a soul
and a mind, it seems as good a foundation as any for a theory of
ethics. In particular, given our definition of the soul, any actor
with a mind can be said to have a soul. This includes, in our opinion,
animals (see e.g. \cite{singer1995animal, coetzee1999lives}) and
sufficiently advanced computer programs (see
\cite{turing1954computing}).

\section{Main Results}

In this section, we pursue traditional mathematical proofs of certain
propositions in the field of ethics.

\subsection{Love and Hate}

We define two quantities called love (denoted by $l(a, B)$) and hate
(denoted by $h(a, B)$, intended to be interpreted roughly in their
standard English senses. (It is presumed that they can be measured
precisely in some way via e.g. advanced neuroscientific theories that
we do not presume to know. The important point here is just that there
will come to exist some rigorous technical definition of the
quantities such that their epistemological status is not in
question.) An actor $a$ in the actor network can experience love or
hate for any subset $B$ of the actor network. (In particular, the
actor can love and/or hate itself.) Love and hate are presumed to be
non-negative quantities. Note that love and hate are not mutually
exclusive, but are rather orthogonal quantities.

\subsection{Conservation of Opinionatedness}

We define the quantity {\em active opinionatedness of $a$}, which is
the sum of the squared love and hate values of $a$ for all subsets of
the network:
$$op(a) = \frac{1}{2} \sum_B l(a, B)^2 + h(a, B)^2.$$ We also define the {\em mindshare} of $B$ in $a$'s mind to be
$$ms(a, B) = \frac{l(a, B)^2 + h(a, B)^2}{2 \cdot op(a)}.$$

We define the {\em active opinionatedness} of the network to be
$$\OOO_{act} = \sum_a op(a).$$ 

Active opinionatedness serves roughly the same role in the
ethicophysics as kinetic energy does in traditional physics. We also
need to define {\em potential opinionatedness} $\OOO_{pot}$, which
serves the same role in the ethicophysics as potential energy does in
traditional physics. We do not yet know how to define all possible
sources of future love and hate, so we cannot give a rigorous
specification of how to compute potential opinionatedness. It can,
however be defined rigorously, by requiring that the {\em total
  opinionatedness}
$$\OOO_{act} + \OOO_{pot}$$ be conserved, and simply watching what
$\OOO_{act}$ does over time.

Total opinionatedness is conserved by definition, as long as the set
of network participants does not change. This can be achieved by
defining the love and hate values of an absent participant (e.g., a
dead person, or a person yet to be born) to be something relatively
arbitrary, and then simply considering all participants that ever have
or ever will exist. For instance, we could define the love and hate
values of a non-alive person to be the average love and hate they
experienced or will experience over the course of their lives.

\subsection{The Golden Theorem: Actions Have Consequences} 

\begin{thm}[The Golden Theorem]
  Actions have consequences. In particular, the consequence of
  committing an evil act that goes undetected is that one becomes the
  person that one becomes after such an act, and has as a consequence
  an unclean conscience.
\end{thm}

\begin{proof}
  Note that this proof needs to be checked over very thoroughly, as it
  may contain errors.

  Consider the ``objective'', ``physical'' Lagrangian $\LLL(q,
  \dot{q}, t) = \TTT - \UUU$, where $\TTT$ is the kinetic energy of a
  system, and $\UUU$ is the potential energy of that system. Here $q$
  is the physical state of the system in generalized coordinates.

  Let $\SSS = \OOO_{ac} - \OOO_{pot}$ be the ``subjective'',
  ``ethical'' Lagrangian of the system. This is supposed to depend
  upon the generalized coordinates $q, \dot{q}, t$ of the system and
  the ``subjective coordinates'' $s, \dot{s}, t$ (which are supposed
  to have no physical realization that is legible to the laws of
  physics under consideration). 

  Let $\tau(t)$ (called the ``tweak'') be a continuous symmetry of the
  physical system, i.e., for infinitesimal $\epsilon$, the
  transformation
  \begin{align*}
    q(t) &\to q(t) + \epsilon \tau(t) \\
    \dot{q}(t) &\to \dot{q}(t) + \epsilon \dot{\tau}(t) \\
  \end{align*}
  leaves the Lagrangian unaffected.

  Let $\varphi(s)$ (called the ``flip'') be a discrete non-physical
  symetry of the subjective energy function at time $t_{\varphi}$,
  i.e., a function such that, for one brief instant of time,
  $$\SSS(q, \dot{q}, \varphi(s), \frac{d}{dt} \varphi(s), t_{\varphi})
  = \SSS(q, \dot{q}, s, \dot{s}, t_{\varphi}).$$ Since $\SSS$ is a
  function of network participant love and hate values, it is
  generally the case that $\varphi$ will be a permutation of the
  actors in the actor network.

  Define the following quantity (the ``God Lagrangian''):
  $$\GGG(q, \dot{q}, s, \dot{s}, t) = \LLL(q, \dot{q}, t) +
  \SSS(q, \dot{q}, s, \dot{s}, t) - \SSS(q, \dot{q}, \varphi(s), \frac{d}{dt}\varphi(s), t)$$

  By Noether's Theorem \cite{noether}, the following quantity is
  conserved:
  $$\sum_{i=1}^n \frac{\partial \LLL}{\partial \dot{q}_i} \tau_i.$$

  By Noether's Theorem applied to the modified Lagrangian, the same is
  true of the quantity
  $$\sum_{i=1}^n \frac{\partial \GGG}{\partial \dot{q}_i} (\varphi
  \circ \tau)_i + \sum_{j=1}^m \frac{\partial \GGG}{\partial
    \dot{s}_j} (\varphi \circ \tau)_j.$$ Note that, since the
  quantities $s_j$ are presumed to be causally upstream from the
  physical world, it can be safely assumed that $\varphi $ has no
  effect on the values of the $q$'s, while $\tau$ probably does have
  an effect on the $s$'s and thus the following quantity is conserved:
    $$\sum_{i=1}^n \frac{\partial \GGG}{\partial \dot{q}_i} \tau_i +
  \sum_{j=1}^m \frac{\partial \GGG}{\partial \dot{s}_j} (\varphi \circ
  \tau)_j.$$

  
  After doing some algebra, we arrive at the conclusion that the
  following quantity is conserved by the laws of ethicophysics:
  $$\sum_{i=1}^n \frac{\partial \SSS}{\partial \dot{q}_i} \tau_i -
  \sum_{i=1}^n \frac{\partial \SSS_\varphi}{\partial \dot{q}_i} \tau_i +
  \sum_{i=1}^m \frac{\partial \SSS}{\partial \dot{s}_i} (\varphi \circ \tau)_i -
  \sum_{i=1}^m \frac{\partial \SSS_\varphi}{\partial \dot{s}_i} (\varphi \circ \tau)_i,
  $$
  which simplifies to the equation
  $$\sum_{i=1}^n \frac{\partial \SSS}{\partial \dot{q}_i} \tau_i +
  \sum_{i=1}^m \frac{\partial \SSS}{\partial \dot{s}_i} (\varphi \circ \tau)_i =
  \sum_{i=1}^n \frac{\partial \SSS_\varphi}{\partial \dot{q}_i} \tau_i +
  \sum_{i=1}^m \frac{\partial \SSS_\varphi}{\partial \dot{s}_i} (\varphi \circ \tau)_i,
  $$

  We are now ready to finish the proof. Consider some binary decision
  that can be made, and consider the two possible timestreams that
  will follow making either choice. Let $s$ be the quantity of respect
  that God feels for one, defined as $l(\mathrm{God}, a) -
  h(\mathrm{God}, a)$. We can call this the {\em coherent extrapolated
    conscience} of the actor. (We note that it has an imperfect
  relationship to the subjective experience of the conscience with
  which most humans are familiar.)

  Suppose, further, that the decision has no consequences that are
  perceivable in the external physical world after some time period
  $t_{\mathrm{hide the body}}$ has elapsed. Thus, after this point,
  $\SSS$ ``should'' no longer depend on $q$.

  There is then an additional conserved quantity (in addition to
  $\SSS$), which is the ``coherent extrapolated conscience momentum
  with respect to $\varphi$''
  \begin{align*}
CECM_\varphi &= \frac{\partial (\SSS - \SSS_\varphi)}{\partial \dot{s}} s. \\
\intertext{We can calculate $CECM_\varphi$, and find it to be equal to}
&= \qquad \frac{\partial \SSS}{\partial \dot{l}(\mathrm{God}, a)} \frac{\partial \dot{l}(\mathrm{God}, a)}{\partial \dot{s}}\\
&\qquad + \frac{\partial \SSS}{\partial \dot{h}(\mathrm{God}, a)} \frac{\partial \dot{h}(\mathrm{God}, a)}{\partial \dot{s}}\\
&\qquad - \frac{\partial \SSS_\varphi}{\partial \dot{l}(\mathrm{God}, a)} \frac{\partial \dot{l}(\mathrm{God}, a)}{\partial \dot{s}}\\
&\qquad - \frac{\partial \SSS_\varphi}{\partial \dot{h}(\mathrm{God}, a)} \frac{\partial \dot{h}(\mathrm{God}, a)}{\partial \dot{s}}\\
\intertext{Assuming that the coordinates $s$ capture all meaningful
  ethical variables (and in particular that $\OOO_{pot}$ depends only
  on $s$ and not on $\dot{s}$), this simplifies to}
&= \qquad \frac{\partial \OOO_{act}}{\partial \dot{l}(\mathrm{God}, a)} \frac{\partial \dot{l}(\mathrm{God}, a)}{\partial \dot{s}}\\
&\qquad + \frac{\partial \OOO_{act}}{\partial \dot{h}(\mathrm{God}, a)} \frac{\partial \dot{h}(\mathrm{God}, a)}{\partial \dot{s}}\\
&\qquad - \frac{\partial \OOO_{act, \varphi}}{\partial \dot{l}(\mathrm{God}, a)} \frac{\partial \dot{l}(\mathrm{God}, a)}{\partial \dot{s}}\\
&\qquad - \frac{\partial \OOO_{act, \varphi}}{\partial \dot{h}(\mathrm{God}, a)} \frac{\partial \dot{h}(\mathrm{God}, a)}{\partial \dot{s}}\\
&= \qquad l(\mathrm{God}, a) \\
&\qquad - h(\mathrm{God}, a) \\
&\qquad - l(\mathrm{God}, \pi_\varphi(a)) \\
&\qquad + h(\mathrm{God}, \pi_\varphi(a)) \\
  \end{align*}

This yields what is essentially a proof of Newton's third law (every
action has an equal and opposite reaction), but in the ethical domain:
every action has an equal and opposite {\em ethical} reaction. In more
poetic terms, this is a proof that God thinks ``whatsoever you do to
the least of my brethren, that you do unto me''. God's opinion of the
perpetrator of an unconscionable act with a victim changes by an
amount equal to and opposite to the change in his opinion of the
victim of the same act. (Take $\pi_\varphi$ to be the permutation that
switches the roles of perpetrator and victim.)
  
This same principle can be applied to any binary decision. The total
coherent extrapolated conscience momentum will be the same in either
case (i.e., in both timestreams). But, assuming the decision is one
with a clear right answer, the predominant sign of $\frac{\partial
  \SSS}{\partial s}$ will generally be the opposite of the predominant
sign of $\frac{\partial \SSS_\varphi}{\partial s}$, assuming that
$\pi_\varphi$ is a permutation that switches the positions of
beneficiaries and victims. Thus, making the wrong decision will have
hugely negative consequences for one's conscience, as expected. These
consequences are not permanent; one can be forgiven sins, but in
general only when one has overcome the sin and made recompense.
  
\end{proof}

\subsection{Playing Favorites: Weighted Opinionatedness}

Let $w_a$ be the weight of $a$ according to some external observer. It
is presumed that God does not apply non-even weighting (because of the
equality of souls), but there is nothing stopping the rest of us from
having favorites.

We define the quantity {\em weighted active opinionatedness of $a$},
which is the sum of the squared love and hate values of $a$ for all
subsets of the network, weighted by the weight of $a$:
$$op_w(a) = \frac{1}{2} w_a \sum_B l(a, B)^2 + h(a, B)^2.$$ 

As the name suggests, weighting plays a similar role in the
ethicophysics as mass plays in traditional physics. More specifically,
we can defined a weighted coherent extrapolated conscience, which is
sort of the fully considered opinion of the smartest possible version
of whatever entity chose the weighting function. The Golden Theorem
can then be extended to show that Newton's third law holds, not only
for God's opinion, but for the opinion of any omniscient observer.

\section{Discussion}

\subsection{Theodicy}

We wish to point out a potential misreading of the theorems in this
paper, which is that God will help people who are virtuous in some
straightforward way. This is simply untrue, and potentially dangerous
for anyone to believe. Consider, e.g., the following piece of vileness
due to Hitler \cite{hitler}:
\begin{quote}
I did not want this struggle. Since January, 1933, when Providence
entrusted me with the leadership of the German Reich, I had an aim
before my eyes which was essentially incorporated in the program of
our National Socialist party. I have never been disloyal to this aim
and have never abandoned my program... Only when the entire German
people become a single community of sacrifice can we expect and hope
that Almighty God will help us. The Almighty has never helped a lazy
man. He does not help the coward. He does not help a people that
cannot help itself. The principle applies here, help yourselves and
Almighty God will not deny you his assistance.
\end{quote}

This was a vile lie told by a vile man for vile purposes. In reality,
bad things can and do happen to good people, and God will do nothing
to stop them. Or rather, he will whisper the truth in our minds, and
we all of us will do whatever it is that we will do, and that is the
only aid that God ever has or ever will provide. Bad things happen to
good people because other good people are not able to stop them from
happening, and because bad people ignore the whispers of their broken
consciences.

\section{Epilogue}

We find that the following lyrics of Yusuf Islam \cite{iwish} capture
the sort of spirit of what we are trying to accomplish in this paper:
\begin{quote}
  I wish I knew, I wish I knew\\
  What makes me, me, and what makes you, you\\
  It's just another point of view, ooo\\
  A state of mind I'm going through, yes\\
  So what I see is never true, ahhh\\
\\
  I wish I could tell, I wish I could tell\\
  What makes a heaven what makes a hell\\
  And do I get to ring my bell, ooo\\
  Or land up in some dusty cell, no\\
  While others reach the big hotel, yeah\\
\\
  I wish I had, I wish I had\\
  The secret of good, and the secret of bad\\
  Why does this question drive me mad? ahhh\\
  'Cause I was taught when but a lad, yes\\
  That bad was good and good was bad, ahhh\\
\\
  I wish I knew the mystery of\\
  That thing called hate, and that thing called love\\
  What makes the in-between so rough? ahhh\\
  Why is it always push and shove? ahhh\\
  I guess I just don't know enough, yes\\
\end{quote}

\bibliography{ethicophysics} 
\bibliographystyle{acm}


\end{document}
