\documentclass{article}
\usepackage{amsmath, amssymb, amsthm, leonine}

\title{Ethicophysics II: Affilliation Economics and Naturalistic Game
Theory}

\author{Eric Purdy}

\begin{document}

\maketitle

\begin{abstract}
\end{abstract}

\section{Introduction}

In this brief document, I lay out the bones of a proposed theory of
human nature suitable for allowing certain useful computations to be
performed. Nothing in this document contradicts the existence of free
will, but its main thrust is to argue that people can be manipulated
somewhat well through the clever use of microtargeted lies, and even
better through the use of misleading microtargeted truths. This is
rather frightening to me, as I am at root a rather honest man. I ask
that anyone reading this use it to fight for truth and justice, but I
realize that this is, at root, a fool's hope. Ultimately, evil wants
what it wants, and it will take what it can.

On the plus side, I think you will find that telling the truth can be
more effective as a tactic than cynics would have you believe.

\section{Affiliation Economics}

We wish to specify a collection of distinct but intersecting
economies. Each economy can be thought of as a game played between two
opposing teams. Or, if you would rather, a war fought between two
opposing armies. We prefer the game terminology, because it helps to
remind us that win-win solutions exist in far more situations than
most people would believe. Ultimately, fundamentally, peace is
possible.

Classical economics is the study of the game between the haves and the
have-nots, between the rich and the poor. I put it to you that it is a
tolearbly complete science, but that it is sorely lacking in its
failure to understand any human value that cannot be quantified in
dollars or utils or what-have-you. Affiliation economics is a sort of
intellectual trick to try to transport the useful parts of economics
to a setting in which the full spectrum of human values can be
observed and reasoned about.

Fortunately for us, most of the concepts necessary to understand
affiliation economics are already well-understood by various
subcultures in society, and by various academic communities. In
general, whenever two opposing forces learn of each other's existence,
the initial result is rather unpleasant, and involves one of the sides
brutally subjugating the other. Such is, unfortunately, the lesson
that history has taught us. This dynamic is what I take to be the
content of Hegel's treatment of the Master-Slave dialectic.

Since this brutal subjugation tends to resolve itself in favor of one
side or the other, at least initially, each game can be thought of as
having a ``natural'' ``top'' side, whose position is stronger. The
advantages that accrue to those on the top side are what is generally
referred to as ``privilege''. The disadvantages that accrue to those
on the bottom side include what is generally referred to as ``stigma''
and ``marginalization''. Essentially, the main activity of most of
those on the top side is reinforcing their power over the bottom side
(``the rich get richer''), while the main activity of most of those on
the bottom side is simply trying to get by under a crooked system.

\section{Continuous Actor Space}

The fundamental problem in affiliation economics that separates it
from classical economics is the necessity of identifying which players
are on which team, and to what extent. (The problem of identifying who
has how much money is, as far as I know, not treated in great depth in
the existing economics literature. There is, of course, a rich
literature on ``signaling'' in classical game theory that one can draw
on.) This is a problem that is so complex and thorny that most people
who have studied it despair of ever sorting out the truth from the
lies. And this worry is merited, but we can at least postulate a
rather simple description of the space of possibilities. We refer to
this postulated space as ``continuous actor space'', because it
describes the space of possible actors that can exist in reality. It
is, of course, an incomplete description; ultimately, each human is a
beautiful and infinite mystery. But I put it to you that certain
large-scale regularities in human nature exist, and are necessary to
explain the human capacity for making any sense whatsoever of the
complex web in which we live.

Continuous actor space is actually quite a simple thing. We can posit
two versions, the bounded and unbounded versions. The bounded version
is simply $[-1, 1]^n$, where $n$ is the number of affiliation
economies we have chosen to include in our model. The unbounded
version is simply $\mathbb{R}^n$. It is, of course, trivial to map
back and forth between the two spaces via the $\mathrm{tanh}$ map, as
is standard in deep learning.

The fundamental problem is then simply the problem of determining
someone's ``true'' place in continuous actor space given observations
of their behavior over time. Since any action taken in public is known
to be in view of others, fundamentally all actions taken in public are
suspect; they function more as ``signals'' in the sense of classical
game theory, and less as any reliable indication of the contents of an
individual soul. We also have extensive evidence that so-called
``moles'' can survive for decades in intelligence services while
performing their roles to an apparently acceptable level; this is
extremely disheartening for anybody hoping to assemble any true
picture of what people are up to. Ultimately, I put it to you, the
only reliable indicators of what is going on inside an individual are as follows:
\begin{itemize}
  \item Longitudinal observations from a very young age (a player can
    only be so good at playing a game for which they do not know the
    ``true'' rules, and such clumsiness reveals some amount about the
    inner workings of the soul)
  \item High-cost actions (i.e., actions which confer no conceivable
    benefit to the actor; this idea is well understood in classical
    game theory and evolutionary game theory)
  \item Total surveillance of every aspect of someone's life, even and
    especially their most private moments
\end{itemize}
Given standard American assumptions about civil liberties, the third
possibility is probably not acceptable to most people. On the other
hand, it is a rather natural result of allowing digital technology
into our homes that organizations like the NSA will acquire such
capabilities unless something rather radical is changed about how
technology is funded and developed.

\section{Main Results}

Ultimately what we are curious about is where each agent is in
continuous actor space. If we know that to any degree of certainty, we
can come down like the fury of God on anyone who accumulates a
position of influence who we do not trust to wield it justly. Or
perhaps rather like a gentle breeze, that whispers in their ear the
way that they can unfuck their soul.

Let us for the sake of exposition consider a 3-d ethicophysics:
good-evil, skilled-unskilled, and rich-poor. Every agent thus has a
good-evil score, a skilled-unskilled score, and a rich-poor
score. (This latter being simply a bank account.)

\subsection{The Desire Field}

Whenever the agent wants something, the desire field comes into
play. It is presumed that we can know which way approximately desire
will pull the soul. In our simple 3-d ethicophysics, desire for
posessions and influence pulls the soul in the evil-skilled-rich
direction. Desire to be of service pulls the soul in the
good-skilled-rich direction. Desire to be lazy pulls the soul in the
poor-skilled direction.  Desire to better oneself pulls the soul in
the good-skilled direction. Desire to increase one's employability
pulls the soul in the rich-skilled direction. We can perhaps identify
other potential desires, but these seem like enough for now.

Let us visualize the results of our simulation as follows: we simply
project dots in a 3-d square representing bounded continuous actor
space and/or a hyperbolic projection of unbounded continuous actor
space. Then we perform a simple market simulation of workers seeking
employment from firms. Firms have internal politics characterized by a
shifting hierarchy DAG that represents who is the "real" boss of
who. It is presumed that all workers are self-interested, and know the
ethicophysics (and thus can reason quite well about the effect of
various choices on their souls and future prospects).

The tricky thing here is that desire pulls the soul, but the position
of the soul shapes desires. This is why the phenomenological tapestry
is such a complex beast. We can simulate this in three alternating
phases: first the position of each soul drives certain actions on the
part of each agent. Then each agent gets some reward from the result
of all actions taken. Then the gradient update from processing the
reward of each agent drives a small update to the position of each
soul. What we are postulating is that the gradient update from
processing a reward will have a particular, predictable effect on the
shape of the soul of the RL agent in question. This is an
experimentally testable hypothesis, possibly.

What game should we choose to experiment with? Let us consider the
following game. Each worker decides how hard to work, on a scale from
-1 (strenuous sabotage) to 0 (apathy, laziness) to +1 (strenuous
service). Each effort number is multiplied by the skill of the
subject. The contributions from each team member are then passed up
the chain to their boss, their boss's boss, etc. We thus wind up with
a total output of the system. This output is presumed to be worth that
amount of reward.

\bibliography{ethicophysics} 
\bibliographystyle{acm}

\end{document}
