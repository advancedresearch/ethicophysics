\documentclass{article}
\usepackage{amsmath, amssymb, amsthm, leonine}

\title{Ethicophysics III: You Say You Want a Revolution?}

\author{Eric Purdy}

\begin{document}

\maketitle

\begin{abstract}
\end{abstract}

\section{Introduction}

In this document, we continue the work begun in Ethicophysics I and
II. In Ethicophysics III we seek to understand the nature of ethics in
something approximating the real world.

As a student in graduate school, the author programmed a system for
distributing jobs over the computer network that was a little too
effective at the job he assigned it. We believe that it provides a
pretty excellent starting point for both attacking a country's
electoral system, and for hardening a country's electoral system such
that no force on Earth could disturb it.

In this document we will attempt to provide two things:
\begin{itemize}
  \item A rigorous mathematical treatment of alienated labor and the
    labor theory of value, sufficient to teach Marxist thinking to
    computers.
  \item A rigorous mathematical treatment of holarchy, a theory due to
    Koestler, sufficient to allow actors to organize collectively
    without very much centralized leadership at all, thus removing the
    key ingredient (the supposed need for a vanguard party authorized
    to commit atrocities in the name of human freedom) that we believe
    fueled the evils of most existing Marxist regimes.
\end{itemize}
Taken together, these two ingredients should allow the execution of a
strategy for revolution that might be termed the {\em anarcho-pacifist
  style of revolution}. This is not a hugely novel contribution, since
it simply validates the tactics of the Arab Spring. Nevertheless, we
hope that providing a rigorous mathematical treatment of the subject
will give hope and courage to those who will need it the most. In
particular, we believe that the aid of computer programmers in
particular will prove key to the struggle, but most of the best
computer programmers do not believe anything they haven't seen a proof
of. Our (probably foolish) hope is that this will contribute to
enabling a series of events worthy of the name ``American Spring''.

\section{On the tripartite nature of signs}

A sign has three components, a signifier (the mark or element in the
world that is taken to signify something), a signified (the content in
the mind of the one who perceives the signification), and a referent
(the thing in external reality that the signifier is taken to refer
to).

It is always questionable whether the referent for a particular
signifier exists in any real sense. What is democracy, for instance?
Does it exist in some Platonic realm of concepts? I put it to you that
this is an incredibly thorny philosophical problem, but that
pragmatist philosophy offers a thread through the maze. Fundamentally,
use is meaning. If people use a signifier for a particular
constellation of uses, then that constellation of uses can be taken as
the referent.

So we can then think of a tripartite graph that is the
phenomenological tapestry. The three kinds of vertices correspond to
signifiers, signifieds, and referents. What is needed to really
mindfuck someone is to know all the signifiers they were exposed to
before a particular time, to guess the signifieds, and to tie them to
a particular set of referents.

\section{Thrasher: on Winning the Influence Game in the Modern World}

The first ingredient is simply an expander graph. (Google it!) An
expander graph is a graph that cannot be cut in half without cutting a
very large number of edges. Basically, what is needed is to inject
ideas into the populace in a way similar to the way that I injected
commands into the computer system at uchicago, so that an all-powerful
adversary who controls the network (and can intercept packets at will)
has to choose between letting in the poison and fucking up the network
so badly that people lose faith in it.

What is the poison? Fundamentally it has to be about inverting
established social hierarchies - ending the Master-Slave dialectic
forever. This was always going to be a tough sell, but it's the only
thing that will do the job that needs doing. Punk rock, y'all.

The real trick, of course, is to grow the network so that it {\em
  becomes an expander graph}. This is the way in which we can prevent
even an all-powerful network administrator from sucking out the poison
that we are injecting.

How do we grow an expander graph? It's a simple matter of using
spectral graph techniques to identify the most problematic cut that
currently exists, and then trying to make edges of love and respect
that cross the cut. Of particular note is that, if one half of the cut
is ``top'' and the other is ``bottom'', i.e., one side has social
superiority, then the most important thing is to craft respect edges
that go from top to bottom, i.e., that go against the usual flow.

\section{Notes}

So now we have a rough sketch of what the rules of the game are, how
it is best played, and how to reshape the board so that we cannot be
stopped. This should be enough to overcome almost any obstacle, if
y'all have the fucking balls to fucking execute on half of this
shit. I will add more details to the various pieces as I think of
them.

\bibliography{ethicophysics} 
\bibliographystyle{acm}


\end{document}
