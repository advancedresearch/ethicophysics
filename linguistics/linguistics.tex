\documentclass{article}

\usepackage{amsmath, amssymb, amsthm}

\title{Langauge as Perceptual Symbol System : towards a synthesis of
  Chomksyan generative linguistics, Pirah\~{a} exceptionalism, and
  modern deep natural language processing, and towards a unified
  theory of language use and language acquisition}

\author{Eric Purdy \\ Your name here}

\begin{document}

In this document, I lay out the bones of a theory of how language
develops in the human mind. This theory depends crucially on the
theoretical work of Noam Chomsky. It also crucially references Richard
Dawkin's ideas about memetic selection. It also relies crucially on
Purdy's theory of askesis, which hopefully you have received a copy of
the paper that lays out that theory. It also relies crucially on the
theory of Pirah\~{a} exceptionalism, due to Daniel L. Everett. It also
relies crucially on the concept of recursive neural networks, as
pioneered by Richard Socher, and exemplified in the paper
(https://nlp.stanford.edu/pubs/SocherBauerManningNg\_ACL2013.pdf) by
Socher, Bauer, Manning, and Ng. Finally, it relies on the argument
made in the paper
(https://www.vicarious.com/2018/02/07/learning-concepts-through-sensorimotor-interactions/)
``Behavior is Everything – Towards Representing Concepts with
Sensorimotor Contingencies'', by Nicholas Hay, Michael Stark, Alexander
Schlegel, Carter Wendelken, Dennis Park, Eric Purdy, and Tom Silver.

Please help me fill in the missing details. You can ask anyone for
help, as long as you add their name to the paper.

\section{Concept Learning Challenges}

We recapitulate here some unpublished work due to Purdy (2015). We
define a number of concept learning challenges that we take to be a
gold standard for the correct use of concepts and language.

We start from a relatively naive view of a concept: a concept has some
denotation (a dictionary definition), and some extension. (I.e., some
set of situations in which adept users of the concept will hold that
the concept is present, and some other set of situation in which adept
users will hold that the concept is absent.)

We posit the following challenges, in rough order of difficulty:

\begin{enumerate}
\item Identify when the concept is present.
\item Identify when the concept is absent.
\item Manipulate one's environment until the concept is present.
  \item Manipulate one's environment to achieve some arbitrary goal,
    all the while maintaining the absence of the concept. E.g., if the concept is ``agent touches red object'', then 
\end{enumerate}

\section{Behavior isn't Everything, but it is a Start}

We note the paper Behavior is Everything. We consider it to be
neo-Skinnerian bunk, on the one hand (it does not pay the requisite
toll to the Cognitive Revolution) and a fundamental contribution in
the history of AI on the other hand (it demonstrates that modern deep
learning systems are fully capable of solving the first two of the
Concept Learning Challenges in a way that is deeply grounded in
sensorimotor experience and that requires absolutely no prior
knowledge).

We thus believe that this work affords the best place to start for a
set of experiments conclusively demonstrating that computers are
capable of using language at a human level. The only thing missing
from the PixelWorld environment is the ability to perform speech
acts. We envision these being produced letter by letter by the agent;
the agent should be able to produce speech acts up to a certain
maximum size on every turn.

In return, the agent should receive speech acts back from the Teacher,
a software component that produces a small list of stock response
speech acts to situations in a dumb, hard-coded manner. The idea is
that the Teacher is a native speaker who has learned a concept by
rote, while the Agent is a native speaker who has learned a concept in
a manner that is deeply grounded in sensorimotor experience.

\section{A theory of Pirah\~{a} Exceptionalism}

I posit that the thing that makes Pirah\~{a} such a powerful and
durable language despite its many, many shortcomings is that it
crucially incorporates a large number of continuous axes into its
phonemic repertoire. In this theory, Pirah\~{a} speakers are sort of
grandmasters of a new way of talking that the rest of us could stand
to learn a thing or two about. (Slash, we already know these things in
our own language, and use them to select tone, intonation, volume, and
other such continuous attributes of our speech acts.)

Basically, let us suppose that each signifier comes with some number
of continuous-valued attributes that the speaker can select in any way
he or she chooses and the listener can interpret in any way he or she
chooses. For the safety and comfort of both listener and speaker, let
us bound all of these continuous values between -1 and 1. We posit
that the interpretation side of this is performed by a mechanism akin
to that of Socher et al.

How is language produced? We believe that it is chosen by a
reinforcement learning agent to further both instrumental and ultimate
goals of the agent. The agent makes use of what we term a Huffman
encoding table with a discriminative post-filter; this is a short
description of what we believe the cerebellum does. We have enclosed
some notes on the cerebellum and Purdy's theory of askesis.

\section{Galifreyan signifier notation}

In this section, I lay out some notation that will hopefully act as a
spur to the intuition, much as Feynman diagrams in physics or Leibniz
notation in calculus do.

We will write English words and other linguistic signifiers in a funny
way depending on how they were produced:

\begin{itemize}
\item $\langle word |$ is how we will write a word produced by the left brain
\item $| word \rangle$ is how we will write a word produced by the right brain
\item $\langle word \rangle$ is how we will write a word produced by
  the two halves of the brain working together. This is the typical
  pattern of an adult speaker of a particular language.

\item $| \widehat{word} |$ is how we will write a word produced
  according to consensus reality and the accepted rules of the
  language in question
\item $\langle \widehat{word} |$ is how we will write a word produced
  by the left brain, in accordance with consensus reality and the
  accepted rules of language
\item $| \widehat{word} \rangle$ is how we will write a word produced
  by the right brain, in accordance with consensus reality and the
  accepted rules of language
\item $\langle \widehat{word} \rangle$ is how we will write a word
  produced by the two halves of the brain working together, in
  accordance with consensus reality and the accepted rules of
  language. Such a word is what is generally called ``true''.
  
\end{itemize}

\section{The Chomsky-Dawkins Law of Memetic Selection}

Now that we have this notation, we can talk about different notions of
truth with much more ease. We also introduce some rules for
manipulating this notation:

The fundamental rule is that two signfiers can be combined into a
larger signifier whenever they have the same ``type'', where type
refers to the bracket notation introduced above.

Thus:

\begin{align}
& \langle Colorless \rangle \langle green \rangle \langle ideas \rangle \langle sleep \rangle \langle furiosly \rangle  \\
& \langle Colorless green \rangle \langle ideas \rangle \langle sleep \rangle \langle furiosly \rangle  \\
& \langle Colorless green ideas \rangle \langle sleep \rangle \langle furiosly \rangle  \\
& \langle Colorless green ideas sleep \rangle \langle furiosly \rangle  \\
& \langle Colorless green ideas sleep furiosly \rangle  \\
\end{align}

The above sentence is generally taken to be purest nonsense, i.e., it
will never mean anything. However, language evolves, and things can
come to mean things that they did not mean initially. If you translate
the above sentence from Chomsky's language of thought to my own, you
get something like the following:

\begin{align}
\langle [Tasteless/bland (to the speaker)/offensive (to the hearer)] French/existentialist/Sartre/schizophrenic/(able to explain anything and everything given enough time) ideas spread rapidly/(like a wildfire) \rangle
\end{align}

I hold that this sentence is actually deeply true, and gives one a
good sense of which ideas will spread and which will not. For
instance, Holocaust denialism and 9/11 trutherism both satisfy the
conditions of the above sentence, and both are incredibly persistent
despite the fact that both are obviously nonsense. Birtherism is
another such idea, I would guess. Perhaps theories about the Kennedy
assassination are another such example. Anyway, my point is that the
sentence is less nonsense than it is some sort of law of memetic
selection. (cf the work of Richard Dawkins, esp. The Selfish Gene) I
have thus taken the liberty of naming it the Chomsky-Dawkins law of
memetic evolution.

\section{But how does this explain the acquistion of language?}

Language is a tool for affecting the world around us. We use this tool
efficiently. Efficiency is achieved using byte-pair encoding. (Google
it!) This byte-pair enocding is accomplished neurally through the
cerebellum, according to Purdy's theory of askesis, which you should
have received a copy of with this document.

\end{document}
